\documentclass[a4,10pt,twocolumn]{article}

\usepackage[margin=1.5cm]{geometry}
\usepackage{hyperref}
\usepackage{titlesec}
\usepackage{graphicx}

% Set heading spacings, please don't change this
\titlespacing\section{0pt}{8pt plus 4pt minus 2pt}{0pt plus 2pt minus 2pt}
\titlespacing\subsection{0pt}{8pt plus 4pt minus 2pt}{0pt plus 2pt minus 2pt}
\titlespacing\subsubsection{0pt}{8pt plus 4pt minus 2pt}{0pt plus 2pt minus 2pt}

\author{
  Belinda Kneubühler (2504756K), Ben Hanmer (2505218H),\\
  Daniels Vasiljevs (2500414V), Shaun Loughery (2193422L),\\
  Zsuzsanna Szugyi (2418750S)}

\title{Puzzle Map}
\date{} % Leave empty

\begin{document}

\maketitle


% ------------------------------------
\section*{Introduction}

This is the template for the MHCI coursework report.


% ------------------------------------
\section*{Understanding The Requirements}

Lorem ipsum dolor sit amet, consectetur adipiscing elit, sed do eiusmod tempor incididunt ut labore et dolore magna aliqua. Ut enim ad minim veniam, quis nostrud exercitation ullamco laboris nisi ut aliquip ex ea commodo consequat. Duis aute irure dolor in reprehenderit in voluptate velit esse cillum dolore eu fugiat nulla pariatur. Excepteur sint occaecat cupidatat non proident, sunt in culpa qui officia deserunt mollit anim id est laborum.


% ------------------------------------
\section*{Concept Generation}
Lorem ipsum dolor sit amet, consectetur adipiscing elit, sed do eiusmod tempor incididunt ut labore et dolore magna aliqua. Ut enim ad minim veniam, quis nostrud exercitation ullamco laboris nisi ut aliquip ex ea commodo consequat. Duis aute irure dolor in reprehenderit in voluptate velit esse cillum dolore eu fugiat nulla pariatur. Excepteur sint occaecat cupidatat non proident, sunt in culpa qui officia deserunt mollit anim id est laborum.

% ------------------------------------
\section*{Initial Prototyping}
Once the inital design ideas were agreed upon, the ideas can then be represented as paper prototypes. Each team member is reponsible with designing their own prototype that represents the previously agreed design choices.

Each team member presented their paper prototypes to the other team members. The paper prototypes were based on an imaginary story of a user on the app. The prototypes were evaluated based on several criteria that determines the usability of the app.

Every team member had decided that the map should be the main focus of the screen. The user should always be able to see where they are in location to a map screen. If other screens were needed to convey information, the user should always be able to easily return to the base map screen.

Swipe motions that were near the edges of the screen were avoided. This is because new smart phones use gesture based navigation. Users would accidentally use the gesture based navigation of their smart phone instead of the application.

A large goal of the application was to not discourage users from continuing to use the app. If the user had went to the wrong location, they should still be rewarded for exploring, but informed that they have reached the wrong destination.

The application should encourage individual users to use the application with their friends. Several features to encourage this include the ability to add friends via a QR code. Riddles can be included that require the QR code of another user to be scanned. Random gifts could be given that can only be sent to friends.

It is important that the user is not frustrated when they are unable to solve a riddle and can not get stuck in the same position. Ideas to counteract this include: providing various different riddles that can be swapped between at the same time; providing hints to the user that will eventually give the direct path to the intended destination and providing the user with the ability to scan AR codes without knowing the associated riddle (incase they accidentally scan the wrong code). Additionally, if the user does scan the wrong associated code, they should be given the option to 'cancel' their decision so that they can try to find the right code again.




% ------------------------------------
\section*{Refined Prototyping}
Lorem ipsum dolor sit amet, consectetur adipiscing elit, sed do eiusmod tempor incididunt ut labore et dolore magna aliqua. Ut enim ad minim veniam, quis nostrud exercitation ullamco laboris nisi ut aliquip ex ea commodo consequat. Duis aute irure dolor in reprehenderit in voluptate velit esse cillum dolore eu fugiat nulla pariatur. Excepteur sint occaecat cupidatat non proident, sunt in culpa qui officia deserunt mollit anim id est laborum.

% ------------------------------------
\section*{Demonstrator Prototype}

Lorem ipsum dolor sit amet, consectetur adipiscing elit, sed do eiusmod tempor incididunt ut labore et dolore magna aliqua. Ut enim ad minim veniam, quis nostrud exercitation ullamco laboris nisi ut aliquip ex ea commodo consequat. Duis aute irure dolor in reprehenderit in voluptate velit esse cillum dolore eu fugiat nulla pariatur. Excepteur sint occaecat cupidatat non proident, sunt in culpa qui officia deserunt mollit anim id est laborum.

% ------------------------------------
\section*{Appendices}

% ------------------------------------
\section*{References}

\end{document}
